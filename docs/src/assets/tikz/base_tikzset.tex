\usepackage{ifthen}
\usepackage{xparse}

\usetikzlibrary{decorations.markings}
\usetikzlibrary{arrows.meta}
\usetikzlibrary{positioning}
\usetikzlibrary{calc}

% \newcommand{\scalefont}[2]{\scalebox{#1}[#2]} % Counter-scale the font
% \newcommand{\scalefont}[2]{\scalebox{#1}[#2]} % Counter-scale the font

\pgfdeclarearrow{
  name = bgarrow,
  parameters = { \the\pgfarrowlength },
  setup code = {
    \pgfarrowssettipend{0\pgfarrowlength}
    \pgfarrowssetlineend{0\pgfarrowlength}
    \pgfarrowssetvisualbackend{.5\pgfarrowlength}
    \pgfarrowssavethe\pgfarrowlength
  },
  drawing code = {
    \pgfpathmoveto{\pgfpoint{0\pgfarrowlength}{0pt}}
    \pgfpathlineto{\pgfpoint{-10pt}{-0.45\pgfarrowlength}}
    \pgfusepathqstroke
  },
  defaults = { length = 5mm }
}

\tikzstyle{port}=[circle, minimum width=1cm, minimum height=1cm, align=center]

\NewDocumentCommand{\drawbg}{m m O{none} O{a} O{none} O{none} O{none}}{
    % Arguments:
    % #1 - start node
    % #2 - end node
    % #3 (optional) - Label type (PQ, TW, or EF), default is none
    % #4 (optional) - label position (a or b), default is a
    % #5 (optional) - subscript for labels, default is none
    % #6 (optional) - marking at the start, default is none
    % #7 (optional) - marking at the end, default is none

    % Define decorations based on optional arguments
    \tikzset{
      addmarkings/.style={
        postaction={
          decorate, 
          decoration={
            markings, 
            mark=at position 0 with {\ifthenelse{\equal{#6}{bar}}{\draw[-] (0,-6pt) -- (0,6pt);}{\relax}},
            mark=at position 1 with {\ifthenelse{\equal{#7}{bar}}{\draw[-] (0,-6pt) -- (0,6pt);}{\relax}}
          }
        }
      }
    }

    \draw[-bgarrow, addmarkings] (#1) -- (#2);

    % Only define and place labels if the correct parameters are provided
    \ifthenelse{\NOT\equal{#3}{none}}{
        \ifthenelse{\equal{#5}{none}}{
            \def\subtext{}
        }{
            \def\subtext{_{#5}}
        }
        \ifthenelse{\equal{#3}{PQ}}{
            \def\effort{P}
            \def\flow{Q}
        }{
            \ifthenelse{\equal{#3}{TW}}{
                \def\effort{\tau}
                \def\flow{\omega}
            }{
                \ifthenelse{\equal{#3}{EF}}{
                    \def\effort{e}
                    \def\flow{f}
                }{}
            }
        }
        \ifthenelse{\equal{#4}{a}}{
            \path (#1) -- (#2) node[midway, above] {\(\effort\subtext\)} node[midway, below] {\(\flow\subtext\)};
        }{
            \path (#1) -- (#2) node[midway, left] {\(\effort\subtext\)} node[midway, right] {\(\flow\subtext\)};
        }
    }{}
}

\NewDocumentCommand{\drawellipsis}{m O{} O{x} O{a}}{
  % Drawing ellipsis
  \ifthenelse{\equal{#3}{x}}{
    % Horizontal orientation
    \foreach \x in {-0.15,0,0.15} {
      \fill[black] ($(#1.west)!0.5!(#1.east)$) + (\x,0) circle (1pt);
    }
    \ifthenelse{\equal{#4}{a}}{
      \node[above=0.2cm of $(#1.west)!0.5!(#1.east)$] {#2};
    }{
      \node[below=0.2cm of $(#1.west)!0.5!(#1.east)$] {#2};
    }
  }{
    % Vertical orientation
    \foreach \y in {-0.15,0,0.15} {
      \fill[black] ($(#1.west)!0.5!(#1.east)$) + (0,\y) circle (1pt);
    }
    \ifthenelse{\equal{#4}{a}}{
      \node[right=0.2cm of $(#1.west)!0.5!(#1.east)$] {#2};
    }{
      \node[left=0.2cm of $(#1.west)!0.5!(#1.east)$] {#2};
    }
  }
}